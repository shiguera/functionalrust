\chapter*{Prólogo}
\addstarredchapter{Prólogo}
\chaptermark{Prólogo}
\label{sec_prologo}

\noindent La programación funcional no es nueva. El lenguaje LISP (LISt Processing) está considerado uno de los primeros lenguajes de alto nivel que se desarrollaron, junto a Fortran y Cobol. Durante mucho tiempo se ha considerado un paradigma de programación demasiado teórico y alejado de la resolución de los problemas habituales en la programación real.

\smallskip

Durante ese tiempo, la Programación Orientada a Objetos (POO) se consideraba el paradigma base necesario para la programación en cualquier lenguaje. Con el tiempo, la POO ha mostrado algunas de sus debilidades, originadas por el mecanismo de la herencia entre tipos. Ya en 1994 ``\textit{The gang of four}'', en su famoso libro ``\textit{Design Patterns: Elements of Reusable Object-Oriented Software}'',  indicaban que había que primar la composición frente a la herencia al construir los programas \citep{gammaDesignPatternsElements1994}. 

\smallskip

En los últimos años, la programación funcional ha adquirido especial relevancia. Los principios que rigen la programación funcional son muy simples. Los principales lenguajes han ido incorporando algunos de los elementos que caracterizan a la programación funcional, como las funciones de orden superior, las closures, la inmutabilidad y otros. Es posible escribir programas con un estilo funcional casi en cualquier lenguaje.

\smallskip

Rust es un lenguaje de programación con solo unos años de vida pero que está teniendo un crecimiento muy importante debido la seguridad del manejo de memoria que proporciona, a la velocidad de ejecución del código generado, a la calidad de las herramientas que proporciona y a las buenas sensaciones que proporciona al programar. Rust no es un lenguaje funcional estricto, pero proporciona suficientes elementos para poder ser utilizado como tal. 

\smallskip

A lo largo del curso, se va a hacer una introducción a la utilización del lenguaje Rust y a los conceptos que sustentan el paradigma de la programación funcional. Se utilizará principalmente el lenguaje Rust, aunque también se mostrarán ejemplos en otros lenguajes.

\smallskip

Al final del curso, el alumno comprenderá los conceptos que subyacen bajo el paradigma de la programación funcional, de forma que podrá utilizar algunos de ellos en el lenguaje que utilice habitualmente. Ademas, conocerá los conceptos fundamentales de la programación con Rust, lo que le permitirá adentrarse en el lenguaje, si lo considera de interés.

%\blankpage
